\documentclass{beamer}
%
% Choose how your presentation looks.
%
% For more themes, color themes and font themes, see:
% http://deic.uab.es/~iblanes/beamer_gallery/index_by_theme.html
%
\mode<presentation>
{
  \usetheme{NYU}      % or try Darmstadt, Madrid, Warsaw, ...
  \usecolortheme{default} % or try albatross, beaver, crane, ...
  \usefonttheme{default}  % or try serif, structurebold, ...
  \setbeamertemplate{navigation symbols}{}
  \setbeamertemplate{caption}[numbered]
} 

\usepackage[english]{babel}
\usepackage[T1]{fontenc}
\usepackage[utf8x]{inputenc}
\usepackage{hyperref}
\begin{document}
\title[EE2227]{CONTROL SYSTEMS}
\subtitle{HOMEWORK-1}
\author{BHUKYA SIDDHU\\ EE18BTECH11004}
\institute{IITH}
\date{\today}
\titlegraphic{\hfill\includegraphics[height=1.5cm]{nyu_shanghai}}



\begin{frame}
  \titlepage
\end{frame}
\item \textbf{Question 33}
\item \textbf{Let the state-space representation of an LTI system be.
\\
\\ $\dot{x(t)}=Ax(t)+Bu(t)$
\\y(t)=Cx(t)+Du(t)
\\A,B,C are matrices, D is scalar, u(t) is input to the system and y(t) is output to the system. let 
$$b1 =\begin{vmatrix}
 0&0&1\\
\end{vmatrix}
$$ 
\\$b1^T=B$
\\and D=0. Find A and C.
\\$H(s)=\dfrac{1}{s^3+3s^2+2s+1}$}
\\\item \textbf{Solution}
\\\item \textbf{STATE MODEL}
\\Let U1(t) and U2(t) are the inputs of the MIMO system and y1(t),y2(t) are the output of the system and x1(t) and x2(t) are the state variables. 
\\so output equation is,
\begin{equation}
    y1(t)=C_{11}\times x1(t)+C_{12}\times x2(t)+d_{11}\times U1(t)+d_{12}\times U2(t)
\end{equation}
\\
\begin{equation}
    y2(t)=C_{21}\times x1(t)+C_{22}\times x2(t)+d_{21}\times U1(t)+d_{22}\times U2(t)
\end{equation}
\\
\[
\begin{bmatrix}
y1(t)\\
y2(t)
\end{bmatrix}
=
\begin{bmatrix}
C_{11}&C_{12}\\
C_{11}&C_{12}\\
\end{bmatrix}\times \begin{bmatrix}
x1(t)\\
x2(t)\\
\end{bmatrix}
+
\begin{bmatrix}
d_{11}&d_{12}\\
d_{11}&d_{12}\\
\end{bmatrix} \times\begin{bmatrix}
U1(t)\\
U2(t)\\
\end{bmatrix}

\]
\\therefore Y(t)=C.X(t)+D.U(t)
\begin{equation}
    \dot{x1(t)}=a_{11}\times x1(t)+a_{12}\times x2(t)+b_{11}\times U1(t)+b_{12}\times U2(tx
\end{equation}
\begin{equation}
    \dot{x2(t)}=a_{21}\times x1(t)+a_{22}\times x2(t)+b_{21}\times U1(t)+b_{22}\times U2(t)
\end{equation}

\begin{document}

\[
\begin{bmatrix}
\dot{x1(t)}\\
\dot{x2(t)}
\end{bmatrix}
=
\begin{bmatrix}
a_{11}&a_{12}\\
a_{11}&a_{12}\\
\end{bmatrix}\times \begin{bmatrix}
x1(t)\\
x2(t)\\
\end{bmatrix}
+
\begin{bmatrix}
b_{11}&b_{12}\\
b_{11}&b_{12}\\
\end{bmatrix} \times\begin{bmatrix}
U1(t)\\
U2(t)\\
\end{bmatrix}

\]
\\$therefore$  $\dot{X(t)}$=A.X(t)+B.U(t)
\\
\\\item \textbf{FINDING TRANSFER FUNCTION}
\\
\\ $So$, $\dot{X(t)}$=A.X(t)+B.U(t) $~be~$ $~equation~$ $~1~$
\\
\\ and  Y(t)=C.X(t)+D.U(t)        $~be~$ $~equation~$ $~2~$
\\
\\
\\ by applying laplace transforms on both sides of equation 1
\\
\\ we get
\\
\\S.X(S)-X(0)=A.X(S)+B.U(S)
\\
\\S.X(S)-A.X(S)=B.U(S)+X(0)
\\
\\(SI-A)X(S)=X(0)+B.U(S)
\\
\\X(S)=X(0)([SI-A])^-1 + B.([SI-A])^-1.U(S)
\\
\\ $~Laplace~$ $~transform~$ $~of~$ $~equation~$ $~2~$ $~and~$ $~sub~$ $X(s)$ 
\\
\\Y(S)=C.X(S)+D.U(S)
\\
\\Y(S)=C.[X(0)([SI-A])^-1 + B.([SI-A])^-1.U(S)]+D.U(S)
\\
\\If X(0)=0
\\
\\ then Y(S)=C.[B.([SI-A])^-1.U(S)]+D.U(S)
\\
\\ \dfrac{Y(S)}{U(S)}=C.[B.([SI-A])^-1]+D=H(S)
\bigskip
\\ As$~$ we$~$ know$~$ that
\bigskip
\\ $$Y(s)=H(s) \times U(s)= (\dfrac{1}{s^3+3s^2+2s+1}) \times U(s)$
\\
\\let X(S)=\dfrac{U(S)}{denominator}
\bigskip
\\ Y(S)=X(S)\times numerotor
\bigskip
\\s^3X(s)+3s^2X(s)+2sX(s)+X(s)=U(S)
\\
\[
\begin{bmatrix}
sx(s)\\
s^2x(s)\\
s^3x(s)
\end{bmatrix}
=
\begin{bmatrix}
0&1&0\\
0&0&1\\
-1&-2&-3
\end{bmatrix}\times \begin{bmatrix}
x(s)\\
sx(s)\\
s^2x(s)
\end{bmatrix}
+
\begin{bmatrix}
0\\
0\\
1
\end{bmatrix} \times U
\]
\\
\\ therfore$~~~~$ A=\begin{bmatrix}
0&1&0\\
0&0&1\\
-1&-2&-3
\end{bmatrix}
\bigskip
\\ Since Y(S)=X(S)\times numerator
\\ therefore Y(S)=X(S);
\\
\\Y=
\begin{bmatrix}
1&0&0
\end{bmatrix}\times \begin{bmatrix}
x(s)\\
sx(s)\\
s^2x(s)
\end{bmatrix} 
\\C=\begin{bmatrix}
1&0&0
\end{bmatrix}
\end{document}

\end{document}
